\documentclass{beamer}

\usepackage[T1,T2A]{fontenc}
\usepackage[utf8]{inputenc}
\usepackage[russian, english]{babel}

\usepackage{amsmath, amsfonts, amsthm, amscd}
%\usepackage{graphics}
\usepackage{graphicx}
\usepackage[matrix,arrow,curve]{xy}
%\usepackage{multicol}

%----------------------------------------------------

\newtheorem{problemR}{Проблема}
\newtheorem{aim}{Цель}


\newtheorem{construction}{Construction}
\newtheorem{problems}{Problems}
\newtheorem{conjecture}{Conjecture}
\newtheorem{question}{Question}

\mode<presentation>
\usetheme{Madrid}

\title{Коммуникационная сложность и ее приложения}
\date{2021}
\author{Дима Случ}


\begin{document}



\begin{frame}
\begin{center}


{\large \scshape
Коммуникационная сложность и ее приложения
}
\url{https://github.com/dmitrysluch/3dRenderer/tree/develop}
\end{center}

\end{frame}
\begin{frame}
\frametitle{Задачи и реализованный функционал}
Обязательные требования:
\begin{itemize}
\item Отрисовка 3d моделей
\item Перспективная проекция
\item Z-Buffer
\item Клиппинг
\item Интерактивное приложение
\end{itemize}
\pause
Дополнительно реализовал:
\begin{itemize}
\item Текстурирование (Билинейная фильтрация)
\item Освещение (Phong/Blinn-Phong)
\item Интерфейс для пользовательских материалов
\end{itemize}
\end{frame}

\begin{frame}
\frametitle{Конвейер}
\begin{itemize}
\item Переход между пространствами
\begin{itemize}
    \item Объекта
    \item Мира
    \item Камеры
    \item Homogenous Cull Space
    \item Экрана
\end{itemize}
\pause
\item Растеризация
\pause
\item Пиксельный шейдер
\pause
\item Тесты (например глубины)
\end{itemize}
\end{frame}
\begin{frame}
\frametitle{Проекция}
    \begin{center}
    \includegraphics[scale=0.2]{proj2.png}
    \end{center}
\end{frame}
% \begin{frame}
% \frametitle{Свет}
%     \begin{itemize}
%         \item Рассеяный свет:\\
%         $\displaystyle color \cdot max(0, (normal, direction))$
%         \pause
%         \item Phong:\\
%         $\displaystyle color \cdot max(0, (reflection, camera)^{power}) I [ (normal, direction) > 0]$\\
%         \pause
%         \item Blinn-Phong:\\
%         $\displaystyle color \cdot max(0, (halfway, normal)^{power})I [ (normal, direction) > 0]$
%     \end{itemize}
% \end{frame}

\begin{frame}
\frametitle{Push vs Pull}
   Pull
   \begin{itemize}
        \item Каждый кадр опрашивается клавиатура и в зависимости от этого изменяются модели
        \item Каждый кадр на экран отрисовывается текущее состояние моделей
        \item Цикл активного ожидания и отрисовки
   \end{itemize}
   Push
   \begin{itemize}
    \item Observable и уведомления
    \item Контроллеры обновляет модели
    \item Обновление модели вызывает перерисовку
    \item Нет активного ожидания
   \end{itemize}
\end{frame}
\begin{frame}
\frametitle{Моя реализация}

    \begin{itemize}
        \item Renderer - низкоуровневый
        \item Kernel и модели - высокоуровневая библиотека
        % \item Framework - декларативный синтаксис для сцены (YAML)
   \end{itemize}
   \pause
   \begin{itemize}
        \item Ядро - владеет моделями - умеет отрисовать сцену
        \item Модели - ссылаются на ядро
        \item Контроллеры - меняют модели - kernel.GetComponent<T>()
        \item Меши и материалы - константны, SceneTransform - proxy
   \end{itemize}
\end{frame}
\end{document}

%-----------------------------------------------------

\begin{frame}
\frametitle{Affine differential scheme}

A differential ring is a ring $R$ with
$\Delta=\{\delta_1,\ldots,\delta_m\}$, $\delta_i\delta_j =
\delta_j\delta_i$

\pause

\begin{center}
\begin{tabular}{p{4cm}p{4cm}}

   Affine $\Delta$-varieties over $K$& Reduced $\Delta$-finitely generated $K$-algebras \\


    \multicolumn{2}{c}{

$$
\xymatrix@R=3pt{
    X\subseteq \mathbb A^n_K&\ar[r]&&K\{X\}\\
    \Max^{\Delta} B&&\ar[l]&B
}
$$
   }

\end{tabular}
\end{center}

\pause

$R$ a ring $\longmapsto$ ($\diffspec R$, $\mathcal O_R$) a locally
ringed space \pause (cannot use $\Spec R$)

\pause

\begin{definition}
\parbox{8cm}{
\begin{tabular}{ll}
\multicolumn{2}{l}{$\varphi \colon \diffspec R\to \bigsqcup_{\frak
q} R_{\frak
q}$ and $\varphi(\frak q)\in R_{\frak q}$}\\
$\varphi$ is regular at $\frak p$ if for all $\frak q\in U$\\
$\varphi(\frak q) = a/b$ in $R_{\frak q}$\\
\multicolumn{2}{l}{$\mathcal O_R(U)$ $=$ regular functions in $U$}\\
$X = \diffspec R$
\end{tabular}
}\parbox{4cm}{
\includegraphics[scale=0.5]{struct_sheaf.jpg}
}
\end{definition}



\end{frame}





\begin{frame}
\frametitle{The problem}

\begin{center}
\begin{tabular}{cc}
    {\bf Commutative case} & {\bf Differential case} \\[5pt]

    Set $\widehat{R}=\mathcal O_R(\Spec R)$& Set $\widehat{R} = \mathcal O_R(\diffspec
    R)$\\[5pt]

    $\iota\colon R \to \widehat{R}$& $\iota\colon R\to \widehat{R}$
    \\[5pt]

\pause

    always isomorphism& not necessarily injective\\

    & not necessarily surjective\\
\end{tabular}
\end{center}

\pause

\begin{conjecture}[Jerald Kovacic]
The map
\begin{align*}
    \iota^*\colon \diffspec \widehat{R} &\to \diffspec R\\
    \frak q& \mapsto \iota^{-1}(\frak q)
\end{align*}
is a homeomorphism.

\end{conjecture}

\end{frame}



\begin{frame}
\frametitle{Reduced schemes}



\begin{center}
\begin{tabular}{cc}
    {\bf Commutative case} & {\bf Differential case} \\[5pt]

    $\varphi\colon \Spec R \to \bigsqcup_{\frak q} k(\frak q)$& $\varphi\colon \diffspec R \to \bigsqcup_{\frak q} k(\frak
    q)$\\[5pt]

    \multicolumn{2}{c}{$\varphi(\frak q) = a/b$ in $k(\frak q)$ and $\mathcal O'_R(U) = $ regular functions in $U$}
    \\[5pt]

\pause

    Set $\widehat{R}' = \mathcal O'_R(\Spec R)$& Set $\widehat{R}' = \mathcal O'_R(\diffspec R)$
    \\[5pt]

    \multicolumn{2}{c}{$\iota_r\colon R\to \widehat{R}'$} \\
\end{tabular}
\end{center}

\pause

\begin{theorem}
$\iota_r\colon R\to D\subseteq \widehat R'$. Then $\iota_r^*\colon
\diffspec D\to \diffspec R$ is a homeomorphism.
\end{theorem}

\pause

\begin{corollary}
The mapping $\iota_r^*\colon \diffspec\widehat{R}' \to \diffspec R$
is a homeomorphism.
\end{corollary}

\end{frame}





\begin{frame}
\frametitle{Keigher rings}

\begin{definition}[Keigher ring]
$I\subseteq R$ is a $\Delta$-ideal $\Rightarrow$ $r(I)$ is a
$\Delta$-ideal.
\end{definition}
\pause

$R$ is a Ritt algebra ($\mathbb Q\subseteq R$) $\Rightarrow$ $R$ is
a Keigher ring. \pause

\begin{theorem}
$R$ is a Keigher ring, and we are given $\iota\colon R\to D\subseteq
\widehat R$. Then
$$
\iota^*\colon \diffspec D\to \diffspec R
$$
 is a
homeomorphism.
\end{theorem}
\pause

\begin{corollary}
$R$ is a Keigher ring. Then
$
\iota^*\colon \diffspec \widehat R \to \diffspec R
$
is a homeomorphism.
\end{corollary}

\end{frame}




\begin{frame}
\frametitle{Improvement}

Ehud Hrushovski:

``$R$ to be a Keigher ring'' is a very strong condition

\pause

\begin{theorem}
$R$ is a differential ring such that $\nil R$ is differential and we
are given $ \iota\colon R\to D\subseteq \widehat R$. Then
$$
\iota^*\colon \diffspec D\to \diffspec R
$$
is a homeomorphism.
\end{theorem}

\pause

\begin{corollary}
$R$ is a differential ring such that $\nil R$ is differential. Then
$$
\iota^*\colon \diffspec \widehat R \to \diffspec R
$$
is a homeomorphism.
\end{corollary}



\end{frame}




\begin{frame}
\frametitle{Iterative derivations}



\begin{definition}
Let $\delta=\{\delta^k\}_{k\geqslant 0}$, $\delta^k\colon R\to R$:
\begin{tabular}{ll}
1) $\delta^0(x)=x$& 3) $\delta^{k}(ab)=\sum_{\mu+\nu=k}\delta^\mu(a)\delta^\nu(b)$\\
2) $\delta^k(a+b)=\delta^k(a)+\delta^k(b)$& 4) $\delta^{k}\delta^m = \binom{k+m}{k}\delta^{k+m}$\\
\end{tabular}
\end{definition}

\pause

A differential ring $R$ is a ring with $\Delta=\{\delta_1,\ldots,
\delta_m\}$, $\delta_i$ are iterative derivations, and
$\delta_i\delta_j = \delta_j\delta_i$.

\pause

\begin{definition}
\parbox{8cm}{
\begin{tabular}{ll}
\multicolumn{2}{l}{$\varphi \colon \diffspec R\to \bigsqcup_{\frak
q} R_{\frak
q}$ and $\varphi(\frak q)\in R_{\frak q}$}\\
$\varphi$ is regular at $\frak p$ if for all $\frak q\in U$\\
$\varphi(\frak q) = a/b$ in $R_{\frak q}$\\
\multicolumn{2}{l}{$\mathcal O_R(U)$ $=$ regular functions in $U$}\\
$X = \diffspec R$
\end{tabular}
}\parbox{4cm}{
\includegraphics[scale=0.5]{struct_sheaf.jpg}
}
\end{definition}

\end{frame}




\begin{frame}
\frametitle{Iterative case}

Set $\widehat R = \mathcal O_R(\diffspec R)$.

\pause

\begin{theorem}
$R$ is a differential ring with iterative derivations and
$$
\iota\colon R\to D\subseteq \widehat R.
$$
Then
$$
\iota^*\colon \diffspec D\to \diffspec R
$$
is a homeomorphism.
\end{theorem}

\pause

\begin{corollary}
$R$ is a differential ring with iterative derivations. Then
$$
\iota^*\colon \diffspec \widehat R \to \diffspec R
$$
is a homeomorphism.
\end{corollary}

\end{frame}


\begin{frame}
\frametitle{The structure sheaf}


\begin{center}
\parbox{4cm}{
$$
\xymatrix@R=3pt{
  R\ar[r]^{\iota}&\widehat{R}\\
  {X}\ar@{<->}[r]^{\iota^*}&\widehat{X}\\
  {\frak p}\ar@{<->}[r]&\widehat{\frak p}\\
}
$$
 }\parbox{6cm}{
\includegraphics[scale=0.5]{two_sheaves.jpg}
}
\end{center}
\pause

\begin{question}
Does $\mathcal O_R$ coincide with $\mathcal O_{\widehat R}$?
\end{question}
\pause

$\mathcal O_{\widehat R,\widehat{ \frak p}}$ could contain more
nilpotent elements than $\mathcal O_{R,\frak p}$ \pause

\begin{fact}
If $R$ is reduced. Then $\mathcal O_R = \mathcal O_{\widehat R}$.
\end{fact}

\end{frame}




\end{document}
